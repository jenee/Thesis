%&LaTeX
% !TEX encoding = UTF-8 Unicode
\documentclass{article}
\usepackage[utf8x]{inputenc}
\usepackage[T1]{fontenc}
\usepackage{textcomp}

\begin{document}
{\large Or, more simply:}

\begin{center}
{\large \textbf{Life }}{\large would }{\large \textbf{be ec-sta-sy}}
\end{center}

{\large This musical emphasis matches the spoken emphasis of the phrase, so it 
is intelligible as a lyric. (Though ecstasy's first syllable doesn't start on the 
first part of beat three, it is still on the first part of beat three, and therefore 
still emphasized. Alternatively, since the first part of beat two didn't have a 
hard stop to it, the emphasis could have rolled over to the second part, ``ec'', 
which does have a hard stop.)}

{\large In contrast, take the second measure: the syllables ``You'', ``me'', and 
``less'' are emphasized in the music. This leads to conflicting musical and spoken 
phrasing:}

\begin{center}
{\large Musical Phrasing: }{\large \textbf{You}}{\large  and }{\large \textbf{me}}{\large  
end}{\large \textbf{less}}{\large ly }

{\large Spoken Phrasing: }{\large \textbf{You}}{\large  and }{\large \textbf{me 
end}}{\large lessly}
\end{center}

{\large The singer is now singing the phrase, syllable by syllable, which they 
think of as syllable-note combinations:}

\begin{center}
{\large YOU and ME end LESS lee}
\end{center}

{\large The singer, for his part, is doing what many singers are taught to do, 
to make it easier to sustain the singing of words that end with unsingable consonants: 
the unsingable consonant is displaced onto the front of the next word. In this 
case, the consonant ``d'' is not singable, so he displaces it onto the next syllable, 
when he can: ``and ME'' becomes ``an dME'', and ``end LESS'' becomes ``en dLESS''. 
So, the singer can effectively think of the sung phrase as:}

\begin{center}
{\large YOU an dME en dLESS lee}
\end{center}

\vspace{12pt}
{\large This doesn't cause confusion for listeners, because they're used to hearing 
it. This does mean, however, that lyric placement does not provide an accurate 
barometer to a listener of where a word actually ends.}

{\large In addition, the singer is singing fudging his vowels, like singers are 
taught to do, so ``and'' and ``end'' sound almost indistinguishable. So, really, 
what listeners are hearing is this:}

\begin{center}
{\large YOU en dME en dLESS lee}
\end{center}

\vspace{12pt}
{\large Now, the listener's brain has to take this syllabic gobbledy- gook, and 
parse it into something useful. They've currently got this mess to deal with (represented 
in SAMPA syllables):}

\begin{center}
{\large \textit{\textbf{ju }}}{\large \textit{En }}{\large \textit{\textbf{dmi 
}}}{\large \textit{En }}{\large \textit{\textbf{dl@s }}}{\large \textit{li}}
\end{center}

\vspace{12pt}
{\large They parse the first part just fine, because the emphases}

\begin{center}
{\large \textbf{you }}{\large and }{\large \textbf{me }}{\large \textit{En }}{\large \textit{\textbf{dl@s 
}}}{\large \textit{li}}
\end{center}

{\large But no one says endLESSly. People say ENDlessly. So, the listeners don't 
recognize it. They have to work with what they have. They already turned one ``En 
d'' into an ``and'', so they do it again:}

\begin{center}
{\large \textbf{you }}{\large and }{\large \textbf{me }}{\large and }{\large \textit{\textbf{l@s 
}}}{\large \textit{li}}
\end{center}

{\large Now, they're just left with LESS lee. And that fits Leslie, a proper noun 
that fits in context and in emphasis placement. So, the final heard lyric is:}

\begin{center}
{\large \textbf{you }}{\large and }{\large \textbf{me }}{\large and }{\large \textbf{Les- 
}}{\large lie}
\end{center}

{\large The misunderstanding can be traced back to improper emphasis placement. 
The songwriter probably didn't even think of that, and now he's stuck: a one-hit-wonder 
with a misunderstood song. We bet that in interview after interview, someone asks 
him who Leslie is. It's probably very frustrating --- especially since he could 
have just moved the word an eight note later, and it would have been understood 
perfectly.}

{\large That's the sort of situation this program is going to help avoid.}

{\large III. FUTURE WORK}

{\large We plan to continue developing and refining the methods through which Melody 
Matcher makes its determinations. Eventually, we plan to use this as an underlying 
framework for an interactive virtual environment where your surroundings are affected 
and created via musical and lyrical input. This should be completed in March 2012, 
with incremental updates discussed on www.melodymatcher.com.}

{\large IV. CONCLUSION}

{\large In this paper, we have discussed at a high level parts of the music-lingustic 
approach that Melody Matcher takes to measure the intelligibility of lyrics. We 
covered some of the major reasons that lyrics get misheard, along with a few examples. 
Melody Matcher's specific implementation details, while fully specified elsewhere, 
were outside the scope of this abstract, and we hope to cover them in a later paper.}

\end{document}
