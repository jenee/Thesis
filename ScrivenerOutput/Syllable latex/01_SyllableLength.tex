\documentclass[10pt,oneside]{memoir}
\usepackage{layouts}[2001/04/29]
\makeglossary
\makeindex

\def\mychapterstyle{default}
\def\mypagestyle{headings}
\def\revision{}

%%% need more space for ToC page numbers
\setpnumwidth{2.55em}
\setrmarg{3.55em}

%%% need more space for ToC section numbers
\cftsetindents{part}{0em}{3em}
\cftsetindents{chapter}{0em}{3em}
\cftsetindents{section}{3em}{3em}
\cftsetindents{subsection}{4.5em}{3.9em}
\cftsetindents{subsubsection}{8.4em}{4.8em}
\cftsetindents{paragraph}{10.7em}{5.7em}
\cftsetindents{subparagraph}{12.7em}{6.7em}

%%% need more space for LoF numbers
\cftsetindents{figure}{0em}{3.0em}

%%% and do the same for the LoT
\cftsetindents{table}{0em}{3.0em}

%%% set up the page layout
\settrimmedsize{\stockheight}{\stockwidth}{*}	% Use entire page
\settrims{0pt}{0pt}

\setlrmarginsandblock{1.5in}{1.5in}{*}
\setulmarginsandblock{1.5in}{1.5in}{*}

\setmarginnotes{17pt}{51pt}{\onelineskip}
\setheadfoot{\onelineskip}{2\onelineskip}
\setheaderspaces{*}{2\onelineskip}{*}
\checkandfixthelayout

\usepackage{fancyvrb}			% Allow \verbatim et al. in footnotes
\usepackage{graphicx}			% To include graphics in pdf's (jpg, gif, png, etc)
\usepackage{booktabs}			% Better tables
\usepackage{tabulary}			% Support longer table cells
\usepackage[utf8]{inputenc}		% For UTF-8 support
\usepackage[T1]{fontenc}		% Use T1 font encoding for accented characters
\usepackage{xcolor}				% Allow for color (annotations)

%\geometry{landscape}			% Activate for rotated page geometry

%\usepackage[parfill]{parskip}	% Activate to begin paragraphs with an empty
								% line rather than an indent


\def\myauthor{Author}			% In case these were not included in metadata
\def\mytitle{Title}
\def\mykeywords{}
\def\mybibliostyle{plain}
\def\bibliocommand{}

\VerbatimFootnotes
\def\myauthor{Jenee Hughes}
\def\baseheaderlevel{1}
\def\format{complete}
\def\mytitle{ThesisDocumentsMelodyMatcher}


%
%	PDF Stuff
%

%\ifpdf							% Removed for XeLaTeX compatibility
%  \pdfoutput=1					% Removed for XeLaTeX compatibility
  \usepackage[
  	plainpages=false,
  	pdfpagelabels,
  	pdftitle={\mytitle},
  	pagebackref,
  	pdfauthor={\myauthor},
  	pdfkeywords={\mykeywords}
  	]{hyperref}
  \usepackage{memhfixc}
%\fi							% Removed for XeLaTeX compatibility


%
% Title Information
%


\ifx\latexauthor\undefined
\else
	\def\myauthor{\latexauthor}
\fi

\ifx\subtitle\undefined
\else
	\addtodef{\mytitle}{}{ \\ \subtitle}
\fi

\ifx\affiliation\undefined
\else
	\addtodef{\myauthor}{}{ \\ \affiliation}
\fi

\ifx\address\undefined
\else
	\addtodef{\myauthor}{}{ \\ \address}
\fi

\ifx\phone\undefined
\else
	\addtodef{\myauthor}{}{ \\ \phone}
\fi

\ifx\email\undefined
\else
	\addtodef{\myauthor}{}{ \\ \email}
\fi

\ifx\web\undefined
	\else
		\addtodef{\myauthor}{}{ \\ \web}
\fi

\title{\mytitle}
\author{\myauthor}

\begin{document}

\chapterstyle{\mychapterstyle}
\pagestyle{\mypagestyle}

%
%		Front Matter
%

\frontmatter


% Title Page

\maketitle
\clearpage

% Copyright Page
\vspace*{\fill}

\setlength{\parindent}{0pt}

\ifx\mycopyright\undefined
\else
	\textcopyright{} \mycopyright
\fi

\revision

\begin{center}
\framebox{ \parbox[t]{1.5in}{\centering Formatted for \LaTeX  \\ 
 by MultiMarkdown}}
\end{center}

\setlength{\parindent}{1em}
\clearpage

% Table of Contents
\tableofcontents
%\listoffigures			% activate to include a List of Figures
%\listoftables			% activate to include a List of Tables


%
% Main Content
%


% Layout settings
\setlength{\parindent}{1em}

\mainmatter
Syllable Length
I had not originally thought of the issue of note-length/weight vs. syllable-length/weight when matching lyrics to music, but came to my attention during the literature review\footnote{G. Lees, The Modern Rhyming Dictionary, Cherry Lane Music, 1987.}, and has since become an integral part of my thesis.
The general idea is that it takes longer to say some sounds than it does others.  For example: We can easily hear that the word `a' takes a lot less time to say than the word ``string'', but they're both single syllable words. 
Turns out, syllables are horrible ways of measuring duration.  They are, however, built of more atomic parts: phonemes. Each syllable contains multiple phonemes, and those phonemes are organized very specifically, as described in the next section.


%
% Back Matter
%

\backmatter
%\appendixpage

%	Bibliography
\bibliographystyle{\mybibliostyle}
\bibliocommand

%	Glossary
\printglossary


%	Index
\printindex

\end{document}
